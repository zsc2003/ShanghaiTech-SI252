\begin{homeworkProblem}

Illustrate the relationship between sampling and optimization.

\solution

1. Convert optimization problem to sampling problem. \\
An optimization problem is usually formulated as
\begin{align*}
\max_{x}  &\quad f(x) \\
\text{s.t.} &\quad f_i(x) \leq 0
\end{align*}
The constrained optimization problem is not suitable for sampling in MCMC. So we can consider applying the penalty to the objective function:
$$\tilde{f}(x) = f(x) + \lambda \cdot f_i(x)\cdot\I_{f_i(x) > 0}$$
where $\lambda > 0$ and $f_i(x)\cdot\I_{f_i(x) > 0}$ measures constraint violations. \\
To get the optimal value, we can apply the Metropolis random walk. For each step, translate from the current state $x$ to a new state $x'$ by
$$x' = x + \epsilon, \quad \epsilon \sim \mathcal{N}(0, \sigma^2 I)$$
with acceptance rate
$$a_{x,x'} = \min\left(1, e^{\beta\left(\tilde{f}(x')-\tilde{f}(x)\right)} \right)$$
$\beta$ could be selected similarly to the Simulated Annealing method. Then the sampled distribution with maximum $\tilde{f}(x)$ could be approximated regarded as the optimal solution. \\

2. Convert sampling problem to optimization problem. \\
Suppose that we have a target distribution $\pi$, which is difficult to sample. We could try to use $q_{\theta}$ to approximate it, where $\theta$ is the parameters. Thus, we can minimize the KL-divergence $KL(q_{\theta}||\pi)$ to let the estimated distribution converge to the target distribution, and then sample from $q_{\theta}$, which is easier to sample. Thus the sampling problem could be converted to the optimization problem.

Usually, the optimization problem use the Variational Inference to approximate the target distribution. i.e. get the evidence lower bound (ELBO) of the KL-divergence, make the optimization problem easier. \\

Reference: \href{https://zhuanlan.zhihu.com/p/88336614}{https://zhuanlan.zhihu.com/p/88336614}

\end{homeworkProblem}

\newpage