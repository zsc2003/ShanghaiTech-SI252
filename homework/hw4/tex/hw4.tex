\documentclass{article}

\usepackage{fancyhdr}
\usepackage{extramarks}
\usepackage{amsmath}
\usepackage{amsthm}
\usepackage{amsfonts}
\usepackage{tikz}
\usepackage{algorithm}
\usepackage{algpseudocode}
\usepackage{enumerate}
\usepackage{tikz}
\usepackage{dsfont}
\usepackage{booktabs, multirow, array} % draw tables
\usepackage{hyperref}
\usepackage{colortbl} % cellcolor
\usetikzlibrary{arrows,automata,positioning}

%
% Basic Document Settings
%

\topmargin=-0.45in
\evensidemargin=0in
\oddsidemargin=0in
\textwidth=6.5in
\textheight=9.0in
\headsep=0.25in

\linespread{1.1}

\pagestyle{fancy}
\lhead{\hmwkAuthorName}
\chead{\hmwkClass : \hmwkTitle}
\rhead{\firstxmark}
\lfoot{\lastxmark}
\cfoot{\thepage}

\renewcommand\headrulewidth{0.4pt}
\renewcommand\footrulewidth{0.4pt}

\setlength\parindent{0pt}

%
% Create Problem Sections
%

\newcommand{\enterProblemHeader}[1]{
    \nobreak\extramarks{}{Problem \arabic{#1} continued on next page\ldots}\nobreak{}
    \nobreak\extramarks{Problem \arabic{#1} (continued)}{Problem \arabic{#1} continued on next page\ldots}\nobreak{}
}

\newcommand{\exitProblemHeader}[1]{
    \nobreak\extramarks{Problem \arabic{#1} (continued)}{Problem \arabic{#1} continued on next page\ldots}\nobreak{}
    \stepcounter{#1}
    \nobreak\extramarks{Problem \arabic{#1}}{}\nobreak{}
}

\newcommand*\circled[1]{\tikz[baseline=(char.base)]{
		\node[shape=circle,draw,inner sep=2pt] (char) {#1};}}


\setcounter{secnumdepth}{0}
\newcounter{partCounter}
\newcounter{homeworkProblemCounter}
\setcounter{homeworkProblemCounter}{1}
\nobreak\extramarks{Problem \arabic{homeworkProblemCounter}}{}\nobreak{}

%
% Homework Problem Environment
%
% This environment takes an optional argument. When given, it will adjust the
% problem counter. This is useful for when the problems given for your
% assignment aren't sequential. See the last 3 problems of this template for an
% example.
%

\newenvironment{homeworkProblem}[1][-1]{
    \ifnum#1>0
        \setcounter{homeworkProblemCounter}{#1}
    \fi
    \section{Problem \arabic{homeworkProblemCounter}}
    \setcounter{partCounter}{1}
    \enterProblemHeader{homeworkProblemCounter}
}{
    \exitProblemHeader{homeworkProblemCounter}
}

%
% Homework Details
%   - Title
%   - Class
%   - Due date
%   - Name
%   - Student ID

\newcommand{\hmwkTitle}{Homework\ \#04}
\newcommand{\hmwkClass}{SI252 Reinforcement Learning}
\newcommand{\hmwkDueDate}{April 20, 2025}
\newcommand{\hmwkAuthorName}{Zhou Shouchen}
\newcommand{\hmwkAuthorID}{2021533042}


%
% Title Page
%

\title{
    \vspace{2in}
    \textmd{\textbf{\hmwkClass:\\  \hmwkTitle}} \\
    \normalsize\vspace{0.1in}\small{Due\ on\ \hmwkDueDate\ at 11:59 p.m.(CST)} \\
	\vspace{4in}
}

\author{
	Name: \textbf{\hmwkAuthorName} \\
	Student ID: \hmwkAuthorID}
\date{}

\renewcommand{\part}[1]{\textbf{\large Part \Alph{partCounter}}\stepcounter{partCounter}\\}

%
% Various Helper Commands
%

% Useful for algorithms
\newcommand{\alg}[1]{\textsc{\bfseries \footnotesize #1}}
% For derivatives
\newcommand{\deriv}[1]{\dfrac{\mathrm{d}}{\mathrm{d}x} (#1)}
% For partial derivatives
\newcommand{\pderiv}[2]{\dfrac{\partial}{\partial #1} (#2)}
% Integral dx
\newcommand{\dx}{\mathrm{d}x}
\newcommand{\dt}{\mathrm{d}t}
\newcommand{\du}{\mathrm{d}u}
\newcommand{\dr}{\mathrm{d}r}
\newcommand{\dS}{\mathrm{d}S}
\newcommand{\dtheta}{\mathrm{d}\theta}
\newcommand{\dphi}{\mathrm{d}\phi}
\newcommand{\domega}{\mathrm{d}\omega}
\newcommand{\dU}{\mathrm{d}U}
\newcommand{\dV}{\mathrm{d}V}


% Alias for the Solution section header
\newcommand{\solution}{\textcolor{blue}{\textbf{Solution}}}
% Probability commands: Expectation, Variance, Covariance, Bias
\newcommand{\E}{\mathbb{E}}
\newcommand{\Var}{\mathrm{Var}}
\newcommand{\Cov}{\mathrm{Cov}}
\newcommand{\Bias}{\mathrm{Bias}}
\newcommand{\Unif}{\operatorname{Unif}}
\newcommand{\Beta}{\operatorname{Beta}}
\newcommand{\Bin}{\operatorname{Bin}}
\newcommand{\Bern}{\operatorname{Bern}}
\newcommand{\Expo}{\operatorname{Expo}}
\newcommand{\FS}{\operatorname{FS}}
\newcommand{\Pois}{\operatorname{Pois}}
\newcommand{\N}{\mathcal{N}}
\newcommand{\A}{\mathcal{A}}
\newcommand{\mS}{\mathcal{S}}
\newcommand{\R}{\mathcal{R}}
\newcommand{\mP}{\mathcal{P}}
\newcommand{\I}{\mathbb{I}}
\newcommand{\indep}{\perp \!\!\! \perp} % independent symbol
\newcommand{\argmin}{\mathop{\rm argmin}}
\newcommand{\argmax}{\mathop{\rm argmax}}
\newcommand{\sgn}{\operatorname{sgn}}
\newcommand{\bpi}{\boldsymbol{\pi}}
\newcommand{\cnt}{\text{count}}


\begin{document}

\maketitle
\pagebreak

\begin{homeworkProblem}

[BH Chapter 11, Problem 2]. Let $X_0, X_1, X_2 \ldots$ be an irreducible Markov chain with state space $\{1,2, \ldots, M\}$. $M \geq 3$, transition matrix $Q=\left(q_{i,j}\right)$, and stationary distribution $\mathbf{s}=\left(s_1, \ldots, s_M\right)$. Let the initial state $X_0$ follow the stationary distribution, i.e., $P\left(X_0=i\right)=s_i$.

(a) On average, how many of $X_0, X_1, \ldots, X_9$ equal 3? (In terms of $\mathbf{s}$; simplify.)

(b) Let $Y_n=\left(X_n-1\right)\left(X_n-2\right)$. For $M=3$, find an example of $Q$ (the transition matrix for the \textit{original} chain $X_0, X_1, \ldots$ ) where $Y_0, Y_1, \ldots$ is Markov, and another example of $Q$ where $Y_0, Y_1, \ldots$ is not Markov. In your examples, make $q_{i,i}>0$ for at least one $i$ and make sure it is possible to get from any state to any other state eventually.

\solution

(a) Let the indicator $\I_i$ donate wheher $X_i=3$, and $N$ be the number of $X_0,\ldots,X_9$ equal to 3. Since the initial state $X_0$ follows the stationary distribution, so $X_1,\ldots,X_9$ also follow the stationary distribution. i.e.
$$P(X_i=3)=s_3, \forall i\in\{0,\ldots,9\}$$
Then we can get that:
$$\E[N] = \E\left[\sum_{i=0}^9 \I_i\right] = \sum_{i=0}^9 \E\left[\I_i\right] = \sum_{i=0}^9 P(X_i=3) = \sum_{i=0}^9 s_3 = 10s_3$$

(b) Since $M=3$, and the relationship between $X_i$ and $Y_i$ is:
\begin{align*}
X_i=1 &\Rightarrow Y_i=0 \\
X_i=2 &\Rightarrow Y_i=0 \\
X_i=3 &\Rightarrow Y_i=2
\end{align*}
Define $g(y)$ be a set of values of $x$ such that $Y=g(X)$. i.e. $g(0)=\{1,2\}, g(2)=\{3\}$.

To let $Y_0,Y_1,\ldots$ be Markov, we need to make sure that
$$P(Y_{n+1}=y_{n+1}|Y_n=y_n,\ldots,Y_0=y_0) = P(Y_{n+1}=y_{n+1}|Y_n=y_n)$$
Since $Y_i$ have $2$ possible values, so we can discuss them in $4$ cases:
\begin{itemize}
\item $Y_{n+1}=2$, $Y_n=2$:
\begin{align*}
&\quad\ P(Y_{n+1}=2|Y_n=2,Y_{n-1}=y_{n-1},\ldots,Y_0=y_0) \\
&= P(X_{n+1}=3|X_n=3,X_{n-1}=x_{n-1},\ldots,X_0=x_0) \\
&= P(X_{n+1}=3|X_n=3) \\
&= P(Y_{n+1}=2|Y_n=2)
\end{align*}

\item $Y_{n+1}=0$, $Y_n=2$:
\begin{align*}
&\quad\ P(Y_{n+1}=0|Y_n=2,Y_{n-1}=y_{n-1},\ldots,Y_0=y_0) \\
&= P(X_{n+1}\in g(0)|X_n=3,X_{n-1}=x_{n-1},\ldots,X_0=x_0) \\
&= P(X_{n+1}=1|X_n=3,X_{n-1}=x_{n-1},\ldots,X_0=x_0) + P(X_{n+1}=2|X_n=3,X_{n-1}=x_{n-1},\ldots,X_0=x_0) \\
&= P(X_{n+1}=1|X_n=3) + P(X_{n+1}=2|X_n=3) \\
&= P(X_{n+1}\in g(0)|X_n=3) \\
&= P(Y_{n+1}=0|Y_n=2)
\end{align*}

\item $Y_{n+1}=2$, $Y_n=0$, combined with LOTP:
\begin{align*}
&\quad\ P(Y_{n+1}=2|Y_n=0,Y_{n-1}=y_{n-1},\ldots,Y_0=y_0) \\
&= \sum_{x_n=1,2} P(X_{n+1}=3|X_n=x_n,X_n\in g(0),\ldots,X_0\in g(y_0))P(X_n=x_n|X_n\in g(0),\ldots,X_0\in g(y_0)) \\
&= \sum_{x_n=1,2} P(X_{n+1}=3|X_n=x_n)P(X_n=x_n|X_n\in g(0),\ldots,X_0\in g(y_0))
\end{align*}
Let $\alpha_1=P(X_n=1|X_n\in g(0),\ldots,X_0\in g(y_0)),\alpha_2=P(X_n=2|X_n\in g(0),\ldots,X_0\in g(y_0))$. So we have
\begin{align*}
\alpha_1+\alpha_2 &= 1 \\
P(Y_{n+1}=2|Y_n=0,Y_{n-1}=y_{n-1},\ldots,Y_0=y_0) &= \alpha_1q_{1,3}+\alpha_2q_{2,3}
\end{align*}
Since $g(0)$ have $2$ elements, so there exists many combinations to make $\alpha_1,\alpha_2$ take different values, but $\alpha_1+\alpha_2$ always holds. However, to make the Markov property holds, we need to make sure that $\alpha_1q_{1,3}+\alpha_2q_{2,3}=P(Y_{n+1}=2|Y_n=0)$, where $P(Y_{n+1}=2|Y_n=0),q_{2,3}$ are constants. Thus it must have
$$\alpha_1(q_{1,3}-q_{2,3})=P(Y_{n+1}=2|Y_n=0)-q_{2,3} \Rightarrow q_{1,3}=q_{2,3}$$

\item $Y_{n+1}=0$, $Y_n=0$, combined with LOTP:
\begin{align*}
&\quad\ P(Y_{n+1}=0|Y_n=0,Y_{n-1}=y_{n-1},\ldots,Y_0=y_0) \\
&= \sum_{x_n=1,2} P(X_{n+1}\in g(0)|X_n=x_n,X_n\in g(0),\ldots,X_0\in g(y_0))P(X_n=x_n|X_n\in g(0),\ldots,X_0\in g(y_0)) \\
&= \sum_{x_n=1,2} P(X_{n+1}\in g(0)|X_n=x_n)P(X_n=x_n|X_n\in g(0),\ldots,X_0\in g(y_0))
\end{align*}
Let $\alpha_1'=P(X_n=1|X_n\in g(0),\ldots,X_0\in g(y_0)),\alpha_2'=P(X_n=2|X_n\in g(0),\ldots,X_0\in g(y_0))$. So we have
\begin{align*}
\alpha_1'+\alpha_2' &= 1 \\
P(Y_{n+1}=0|Y_n=0,Y_{n-1}=y_{n-1},\ldots,Y_0=y_0) &= \alpha_1'\left(q_{1,1}+q_{1,2}\right)+\alpha_2'\left(q_{2,1}+q_{2,2}\right)
\end{align*}
Similarly to the analysis above, to make the Markov property holds, it has
$$q_{1,1}+q_{1,2}=q_{2,1}+q_{2,2}$$
So above all, if $q_{1,3}=q_{2,3}$ and $q_{1,1}+q_{1,2}=q_{2,1}+q_{2,2}$, then $Y_0,Y_1,\ldots$ is Markov. \\
And example of $Q$ where $Y_0,Y_1,\ldots$ is Markov is:
$$Q=\begin{pmatrix}
\frac{1}{3} & \frac{1}{3} & \frac{1}{3} \\
\frac{1}{3} & \frac{1}{3} & \frac{1}{3} \\
\frac{1}{3} & \frac{1}{3} & \frac{1}{3}
\end{pmatrix}$$
Examplt of $Q$ where $Y_0,Y_1,\ldots$ is not Markov is:
$$Q=\begin{pmatrix}
\frac{1}{2} & \frac{1}{2} & 0 \\
\frac{1}{3} & \frac{1}{3} & \frac{1}{3} \\
\frac{1}{3} & \frac{1}{3} & \frac{1}{3}
\end{pmatrix}$$

\end{itemize}

\end{homeworkProblem}

\newpage
\begin{homeworkProblem}

\textbf{Student MDPs}

(a) Given the equal option policy, reproduce the state values \& state-action values for student MDP with both theoretical method and iterative policy evaluation method. Then discuss the pros and cons of each method.
\begin{figure}[h]
    \centering
    \includegraphics[width=0.5\textwidth]{./figure/MDP1.png}
\end{figure}

(b) Reproduce the optimal state values, optimal state-action values, and optimal policy for student MDP with theoretical method, policy iteration method and value iteration method. Then discuss the pros and cons of each method.
\begin{figure}[h]
    \centering
    \includegraphics[width=0.5\textwidth]{./figure/MDP2.png}\
    \vspace{-0.5cm}
\end{figure}

\solution

We can number the states for convience, written on the figure with purple $s_1,\cdots,s_4$(The terminal state is ignored, whose state value and action-state values are all $0$). And the action space is $\A=$\{Facebook, Quit, Study, Sleep, Pub\}. And the rewards and transition probabilities are given in the figure. And the discount factor is $\gamma=1$. The codes for (a), (b) could be found in the `hw5\_code.ipynb' file.

(a) From the Bellman Expectation Equations for MDPs, we can get that
\begin{align*}
v_{\pi}(s) &= \sum_{a \in \A}\pi(a|s)\left(\R_s^a + \gamma \sum_{s' \in \mS} \mP_{s,s'}^a v_{\pi}(s')\right) \\
q_{\pi}(s,a) &= \R_s^a + \gamma \sum_{s' \in \mS} \mP_{s,s'}^a v_{\pi}(s')
\end{align*}
1. Theoretical method: \\
We can use the inplace method to solve the state values $v_{\pi}(s)$, i.e. solve the equations:
\begin{align*}
\qquad&\begin{cases}
v_{\pi}(s_1) = 0.5 * (\R_{s_1}^{\text{Facebook}} + 1 * 1 * v_{\pi}(s_1)) + 0.5 *(\R_{s_1}^{\text{Quit}} + 1 * 1 * v_{\pi}(s_2)) \\
v_{\pi}(s_2) = 0.5 * (\R_{s_2}^{\text{Facebook}} + 1 * 1 * v_{\pi}(s_1)) + 0.5 *(\R_{s_2}^{\text{Study}} + 1 * 1 * v_{\pi}(s_3)) \\
v_{\pi}(s_3) = 0.5 * (\R_{s_3}^{\text{Study}} + 1 * 1 * v_{\pi}(s_4)) + 0.5 *(\R_{s_3}^{\text{Sleep}} + 1 * 1 * 0) \\
v_{\pi}(s_4) = 0.5 * (\R_{s_4}^{\text{Study}} + 1 * 1 * 0) + 0.5 * \left(\R_{s_4}^{\text{Pub}} + 1 * \left(0.2 * v_{\pi}(s_2) + 0.4 * v_{\pi}(s_3) + 0.4 * v_{\pi}(s_4)\right)\right)
\end{cases} \\
\Rightarrow\ &\begin{cases}
v_{\pi}(s_1) - v_{\pi}(s_2) = -1 \\
-v_{\pi}(s_1) + 2v_{\pi}(s_2) - v_{\pi}(s_3) = -3 \\
2v_{\pi}(s_3) - v_{\pi}(s_4) = -2 \\
-v_{\pi}(s_2) - 2v_{\pi}(s_3) + 8v_{\pi}(s_4) = 55
\end{cases} \\
\Rightarrow\ &\begin{cases}
v_{\pi}(s_1) = -2.3076923076923066 \\
v_{\pi}(s_2) = -1.3076923076923066 \\
v_{\pi}(s_3) = 2.6923076923076934 \\
v_{\pi}(s_4) = 7.384615384615385
\end{cases}
\qquad\approx\qquad\begin{cases}
v_{\pi}(s_1) = 2.3 \\
v_{\pi}(s_2) = -1.3 \\
v_{\pi}(s_3) = 2.7 \\
v_{\pi}(s_4) = 7.4
\end{cases}
\end{align*}

With calculated state values, we can get the state-action values:
$$\begin{cases}
q_{\pi}(s_1, \text{Facebook}) &= -3.3076923076923066 \\
q_{\pi}(s_1, \text{Quit}) &= -1.3076923076923066 \\
q_{\pi}(s_2, \text{Facebook}) &= -3.3076923076923066 \\
q_{\pi}(s_2, \text{Study}) &= 0.6923076923076934 \\
q_{\pi}(s_3, \text{Study}) &= 5.384615384615385 \\
q_{\pi}(s_3, \text{Sleep}) &= 0 \\
q_{\pi}(s_4, \text{Study}) &= 10 \\
q_{\pi}(s_4, \text{Pub}) &= 4.76923076923077
\end{cases} \qquad\approx\qquad \begin{cases}
q_{\pi}(s_1, \text{Facebook}) &= -3.3 \\
q_{\pi}(s_1, \text{Quit}) &= -1.3 \\
q_{\pi}(s_2, \text{Facebook}) &= -3.3 \\
q_{\pi}(s_2, \text{Study}) &= 0.7 \\
q_{\pi}(s_3, \text{Study}) &= 5.4 \\
q_{\pi}(s_3, \text{Sleep}) &= 0 \\
q_{\pi}(s_4, \text{Study}) &= 10 \\
q_{\pi}(s_4, \text{Pub}) &= 4.8
\end{cases}$$

2. Iterative Policy Evaluation Method: \\
The threshold is set to be $\epsilon=10^{-3}$, after 26 iterations, the state values convergence, and the results are
$$\begin{cases}
v_{\pi}(s_1) = -2.3116692858874535 \\
v_{\pi}(s_2) = -1.3102871136591983 \\
v_{\pi}(s_3) = 2.6919968668115835 \\
v_{\pi}(s_4) = 7.384101760095685
\end{cases}
\qquad\approx\qquad\begin{cases}
v_{\pi}(s_1) = 2.3 \\
v_{\pi}(s_2) = -1.3 \\
v_{\pi}(s_3) = 2.7 \\
v_{\pi}(s_4) = 7.4
\end{cases}$$
The correspondence state-action values are
$$\begin{cases}
q_{\pi}(s_1, \text{Facebook}) &= -3.3116692858874535 \\
q_{\pi}(s_1, \text{Quit}) &= -1.3102871136591983 \\
q_{\pi}(s_2, \text{Facebook}) &= -3.3116692858874535 \\
q_{\pi}(s_2, \text{Study}) &= 0.6919968668115835 \\
q_{\pi}(s_3, \text{Study}) &= 5.384101760095685 \\
q_{\pi}(s_3, \text{Sleep}) &= 0 \\
q_{\pi}(s_4, \text{Study}) &= 10 \\
q_{\pi}(s_4, \text{Pub}) &= 4.768382028031068
\end{cases} \qquad\approx\qquad \begin{cases}
q_{\pi}(s_1, \text{Facebook}) &= -3.3 \\
q_{\pi}(s_1, \text{Quit}) &= -1.3 \\
q_{\pi}(s_2, \text{Facebook}) &= -3.3 \\
q_{\pi}(s_2, \text{Study}) &= 0.7 \\
q_{\pi}(s_3, \text{Study}) &= 5.4 \\
q_{\pi}(s_3, \text{Sleep}) &= 0 \\
q_{\pi}(s_4, \text{Study}) &= 10 \\
q_{\pi}(s_4, \text{Pub}) &= 4.8
\end{cases}$$

3. Pros and cons: \\
Theoretical method calculates the optimal state values $v_*(s)$ by solving a linear system, and its advantage lies in providing an exact solution in one go. It constructs the matrix $\left(I-\gamma\mP^{\pi}\right)^{-1}\R^{\pi}$ to solve for the optimal state values, avoiding the need for iterative processes. The result is accurate, and when the model is  relatively small in scale, the complexity is relatively low. \\
However, the disadvantage of this method is that getting inverse matrix requires $O(n^3)$ time complexity, where $n$ is the number of states. When the state space and action space are large, the theoretical method may become computationally expensive and inefficient. \\

Iterative policy evaluation method iteratively updates the state value function to approximate the optimal solution. It can gradually converge to the optimal solution it can still effectively perform evaluations when the spaces are large. It may not be the accurate solution, but close to the accurate one and computational friendly. \\
However, its disadvantage is that it may have cases that converges slowly. When chasing for a more accurate solution, i.e. setting $\epsilon$ to be smaller, it may converge much slower. \\

(b) Let $v_*(s)$ be the optimal state value function, and $q_*(s,a)$ be the optimal state-action value function. The Bellman Optimality Equation is:
\begin{align*}
v_*(s) &= \max_{a \in \A}\left(\R_s^a + \gamma \sum_{s' \in \mS} \mP_{s,s'}^a v_*(s')\right) \\
q_*(s,a) &= \R_s^a + \gamma \sum_{s' \in \mS} \mP_{s,s'}^a v_*(s') \\
\pi_*(a|s) &= \begin{cases}
1 & \text{if } a\in \argmax\limits_{a\in\A}q_*(s,a) \\
0 & \text{otherwise}
\end{cases}
\end{align*}
Thus, we could get the optimal state value function $v_*(s)$ with different methods, then get optimal state-action value function $q_*(s,a)$ using $v_*(s)$ and final get the optimal policy $\pi^*(a|s)$ using $q_*(s,a)$.

1. Theoretical method: \\
We can re-write the definition of $v_*(s)$ into the an inequality form:
\begin{align*}
v_*(s) = \max_{a \in \A}\left(\R_s^a + \gamma \sum_{s' \in \mS} \mP_{s,s'}^a v_*(s')\right)
&\Rightarrow v_*(s) \geq \R_s^a + \gamma \sum_{s' \in \mS} \mP_{s,s'}^a v_*(s')\ \ \forall a\in A \\
&\Rightarrow -v_*(s) + \gamma \sum_{s' \in \mS} \mP_{s,s'}^a v_*(s') \leq -\R_s^a\ \ \forall a\in A
\end{align*}
So the Bellman Optimality Equation could be re-written into a set of inequality constrains. As we known of the basic knowledge of convex optimization, the optimal solution must be lying on the inconstraint set for the Linear Programming problem, thus we can add a linear objective function to ensure the inequality constrains try to be equal. So the final LP problem is:
\begin{align*}
\min_{v} &\qquad\quad \sum_{s \in \mS} v(s) \\
\text{s.t.} &\ -v(s) + \gamma \sum_{s' \in \mS} \mP_{s,s'}^a v(s') \leq -\R_s^a\ \ \forall s\in\mS, a\in \A
\end{align*}
Thus, solve the LP problem with optimizers, we can get the theoritical optimal state values, optimal state-action values and optimal policy. The constructed LP is:
\begin{align*}
\min_{v} &\qquad\quad \sum_{s \in \mS} v(s) \\
\text{s.t.} &\ \begin{pmatrix}
0 & 0 & 0 & 0 \\
-1 & 1 & 0 & 0 \\
1 & -1 & 0 & 0 \\
0 & -1 & 1 & 0 \\
0 & 0 & -1 & 0 \\
0 & 0 & -1 & 0 \\
0 & 0.2 & 0.4 & -0.6 \\
0 & 0 & 0 & -1
\end{pmatrix}
\begin{pmatrix}
v(s_1) \\ v(s_2) \\ v(s_3) \\ v(s_4)
\end{pmatrix} \leq \begin{pmatrix}
1 \\ 0 \\ 1 \\ 2 \\ 0 \\ 2 \\ -1 \\ -10
\end{pmatrix}
\end{align*}
The result is:
$$\begin{cases}
v_*(s_1) = 6 \\
v_*(s_2) = 6 \\
v_*(s_3) = 8 \\
v_*(s_4) = 10
\end{cases} \qquad \Rightarrow \qquad \begin{cases}
q_*(s_1, \text{Facebook}) &= 5 \\
q_*(s_1, \text{Quit}) &= 6 \\
q_*(s_2, \text{Facebook}) &= 5 \\
q_*(s_2, \text{Study}) &= 6 \\
q_*(s_3, \text{Study}) &= 8 \\
q_*(s_3, \text{Sleep}) &= 0 \\
q_*(s_4, \text{Study}) &= 10 \\
q_*(s_4, \text{Pub}) &= 9.4
\end{cases} \qquad \Rightarrow \qquad \begin{cases}
\pi_*(s_1) = \text{Quit} \\
\pi_*(s_2) = \text{Study} \\
\pi_*(s_3) = \text{Study} \\
\pi_*(s_4) = \text{Study}
\end{cases}$$

2. Policy Iteration Method: \\
We can applying policy iteration method through the following pseudocode to get $v_*(s)$ and $\pi_*(s)$, then get $q_*(s,a)$ by Bellman Optimality Equation using $v_*(s)$: \\
\begin{figure}[!htbp]
    \centering
    \vspace{-0.5cm}
    \includegraphics[width=0.55\textwidth]{./figure/policy_iteration_pseudocode}
    \vspace{-0.5cm}
\end{figure}
The result is:
$$\begin{cases}
v_*(s_1) = 6 \\
v_*(s_2) = 6 \\
v_*(s_3) = 8 \\
v_*(s_4) = 10
\end{cases} \qquad \Rightarrow \qquad \begin{cases}
q_*(s_1, \text{Facebook}) &= 5 \\
q_*(s_1, \text{Quit}) &= 6 \\
q_*(s_2, \text{Facebook}) &= 5 \\
q_*(s_2, \text{Study}) &= 6 \\
q_*(s_3, \text{Study}) &= 8 \\
q_*(s_3, \text{Sleep}) &= 0 \\
q_*(s_4, \text{Study}) &= 10 \\
q_*(s_4, \text{Pub}) &= 9.4
\end{cases} \qquad \Rightarrow \qquad \begin{cases}
\pi_*(s_1) = \text{Quit} \\
\pi_*(s_2) = \text{Study} \\
\pi_*(s_3) = \text{Study} \\
\pi_*(s_4) = \text{Study}
\end{cases}$$
It totally run policy iteration 2 times, and the policy evaluation used 24+4=30 iterations.

3. Value Iteration Method: \\
We can applying policy iteration method through the following pseudocode to get $v_*(s)$ and $q_*(s,a)$, then get $\pi_*(s)$ through $q_*(s,a)$:
\begin{figure}[!htbp]
    \centering
    \includegraphics[width=0.7\textwidth]{./figure/value_iteration_pseudocode}
\end{figure}

It totally run value iteration 4 times. The result is:
$$\begin{cases}
v_*(s_1) = 6 \\
v_*(s_2) = 6 \\
v_*(s_3) = 8 \\
v_*(s_4) = 10
\end{cases} \qquad \Rightarrow \qquad \begin{cases}
q_*(s_1, \text{Facebook}) &= 5 \\
q_*(s_1, \text{Quit}) &= 6 \\
q_*(s_2, \text{Facebook}) &= 5 \\
q_*(s_2, \text{Study}) &= 6 \\
q_*(s_3, \text{Study}) &= 8 \\
q_*(s_3, \text{Sleep}) &= 0 \\
q_*(s_4, \text{Study}) &= 10 \\
q_*(s_4, \text{Pub}) &= 9.4
\end{cases} \qquad \Rightarrow \qquad \begin{cases}
\pi_*(s_1) = \text{Quit} \\
\pi_*(s_2) = \text{Study} \\
\pi_*(s_3) = \text{Study} \\
\pi_*(s_4) = \text{Study}
\end{cases}$$

4. Pros and cons: \\
Theoretical method: \\
Advantage: It can get an exact and optimal solution by formulating the Bellman Optimality Equation as a set of linear inequalities, then solving the problem using linear programming. This method guarantees that the resulting state values and policies are optimal because it directly solves the problem mathematically without iteration. It is highly efficient for small-scale problems. \\
Disadvantages: It is not suitable for large-scale problems as solving the optimization problem requires a much more time, which is not efficient. \\

Policy Iteration: \\
Advantage: By alternating between policy evaluation and policy improvement, the optimal policy can be found and the corresponding optimal state value $V_x (s) $can be calculated, resulting in high accuracy for policy evaluation. \\
Disadvantages: Requires repeated strategy evaluation and improvement, slow convergence speed, especially when the state space is large. \\

Value Iteration: \\
Advantage: By continuously updating the state value function, it is possible to directly approximate the optimal state value $V_x (s) $. It is based on greedy selection, calculating the optimal action only at last. \\
Disadvantage: Due to updating the values of all states at each iteration, it may require more iterations to converge when the state space is large.

\end{homeworkProblem}

\newpage
\begin{homeworkProblem}

[BH Chapter 11, Problem 6]. Daenerys has three dragons: Drogon, Rhaegal, and Viserion. Each dragon independently explores the world in search of tasty morsels. Let $X_n, Y_n, Z_n$ be the locations at time $n$ of Drogon, Rhaegal, Viserion respectively, where time is assumed to be discrete and the number of possible locations is a finite number $M$. Their paths $X_0, X_1, X_2 \ldots$; $Y_0, Y_1, Y_2, \ldots ;$ and $Z_0, Z_1, Z_2, \ldots$ are independent Markov chains with the same stationary distribution $\mathbf{s}$. Each dragon starts out at a random location generated according to the stationary distribution.

(a) Let state 0 be home (so $s_0$ is the stationary probability of the home state). Find the expected number of times that Drogon is at home, up to time 24, i.e., the expected number of how many of $X_0, X_1, \ldots, X_{24}$ are state 0 (in terms of $s_0$ ).

(b) If we want to track all 3 dragons simultaneously, we need to consider the vector of positions, $\left(X_n, Y_n, Z_n\right)$. There are $M^3$ possible values for this vector; assume that each is assigned a number from 1 to $M^3$, e.g., if $M=2$ we could encode the statex $(0,0,0),(0,0,1),(0,1,0), \ldots,(1,1,1)$ as $1,2,3, \ldots, 8$ respectively. Let $W_n$ be the number between 1 and $M^3$ representing ($X_n, Y_n, Z_n$). Determine whether $W_0, W_1, \ldots$ is a Markov chain.

(c) Given that all 3 dragons start at home at time 0, find the expected time it will take for all 3 to be at home again at the same time.

\solution

(a) Let the indicator $\I_i$ denote whether $X_i=0$, and $N$ be the number of $X_0,\ldots,X_{24}$ equal to 0. Since the initial state $X_0$ follows the stationary distribution, so $X_1,\ldots,X_{24}$ also follow the stationary distribution. i.e.
$$P(X_i=0)=s_0, \forall i\in\{0,\ldots,24\}$$
Then we can get that:
$$\E[N] = \E\left[\sum_{i=0}^{24} \I_i\right] = \sum_{i=0}^{24} \E\left[\I_i\right] = \sum_{i=0}^{24} P(X_i=0) = \sum_{i=0}^{24} s_0 = 25s_0$$

(b) $W_i=w_i$ and $(X_i,Y_i,Z_i)=(x_i,y_i,z_i)$ correspond one-to-one, so
\begin{align*}
&\quad\ P(W_{n+1}=w_{n+1}|W_0=w_0,\ldots,W_n=w_n) \\
&= P(X_{n+1}=x_{n+1},Y_{n+1}=y_{n+1},Z_{n+1}=z_{n+1}|X_n=x_n,\ldots,X_0=x_0,Y_n=y_n,\ldots,Y_0=y_0,Z_n=z_n,\ldots,Z_0=z_0) \\
&= P(X_{n+1}=x_{n+1}|X_n=x_n,\ldots,X_0=x_0)P(Y_{n+1}=y_{n+1}|Y_n=y_n,\ldots,Y_0=y_0)P(Z_{n+1}=z_{n+1}|Z_n=z_n,\ldots,Z_0=z_0) \\
&= P(X_{n+1}=x_{n+1}|X_n=x_n)P(Y_{n+1}=y_{n+1}|Y_n=y_n)P(Z_{n+1}=z_{n+1}|Z_n=z_n) \\
&= P(X_{n+1}=x_{n+1},Y_{n+1}=y_{n+1},Z_{n+1}=z_{n+1}|X_n=x_n,Y_n=y_n,Z_n=z_n) \\
&= P(W_{n+1}=w_{n+1}|W_n=w_n)
\end{align*}
The second equality and the second to last equality hold due to the the paths are independent markov chains. So $W_0, W_1, \ldots$ is a Markov chain.

(c) Let $N_t$ be the number of times that all 3 dragons are at home from time $1$ to time $t$. And let $T_i$ be the time interval between the $(i-1)$-th time that all 3 dragons are at home and the $i$-th. Here we define that $X_0=0$ is the $0$-th time that all 3 dragons are at home. Let $T$ be the first time all 3 dragons are at home. From the Markov property, we can get that $T, T_1, \ldots$ are i.i.d. random variables. From the large number of law, we have
$$\E(T)=\lim_{n\to\infty}\dfrac{T_1+\ldots+T_n}{n}=\lim_{t\to+\infty}\dfrac{T_1+\ldots+T_{N_t}}{N_t}$$
We can also get that
\begin{align*}
T_1+\ldots+T_{N_{t}} &\leq t \leq T_1+\ldots+T_{N_{t+1}} \\
\implies \lim_{t\to+\infty}\dfrac{T_1+\ldots+T_{N_{t}}}{N_{t}} &\leq \lim_{t\to+\infty}\dfrac{t}{N_{t}} \leq \lim_{t\to+\infty}\dfrac{T_1+\ldots+T_{N_{t+1}}}{N_{t+1}}
\end{align*}
Since $\lim\limits_{t\to+\infty}\dfrac{T_1+\ldots+T_{N_{t}}}{N_{t}}=\E(T)$, and $\lim\limits_{t\to+\infty}\dfrac{T_1+\ldots+T_{N_{t+1}}}{N_{t+1}}=\E(T)$, we have
$$\lim_{t\to+\infty}\dfrac{t}{N_t}=\E(T)$$
From the property as the stationary distribution of the Markov chain, we have the stationary distribution of $W_0, W_1, \ldots$ has
$$P(W_i=0)=P(X_i=0)P(Y_i=0)P(Z_i=0)=s_0^3$$
Thus we have
\begin{align*}
s_0^3=\lim_{t\to+\infty}\dfrac{N_t}{t} &= \dfrac{1}{\E(T)} \\
\implies \E(T) &= \dfrac{1}{s_0^3}
\end{align*}

So above all, the expected time for all 3 dragons to be at home again at the same time is $\dfrac{1}{s_0^3}$.

\end{homeworkProblem}

\newpage

\end{document}