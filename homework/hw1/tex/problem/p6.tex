\begin{homeworkProblem}

Given a random variable $X \sim \N(0,1)$, evaluate the tail probability $c=P(X>8)$ by Monte Carlo methods with \& without importance sampling. Discuss the pros and cons of importance sampling.

\solution

1. Without importance sampling. \\
The total $10^9$ samples from $\N(0,1)$ have no single sample greater than 8. \\
So the tail probability estimatied is $c=P(X>8)=0$. \\

2. With importance sampling. \\
To calculate $c=P(Y>8)$, where $Y\sim \N(0,1)$. \\
So the PDF of $Y$ is that $f(y)=\dfrac{1}{\sqrt{2\pi}}e^{-\frac{1}{2}y^2}$. \\
It may be difficult to calculate the exact value of $c$ using simple sampling methods (because of the 3$\sigma$'s principle, the result must be very small). \\
So we can use the importance sampling. \\
Take $g\sim \N(8,1)$. \\
Let $Y_1,\cdots,Y_N\sim g$, so the PDF of $g$ is that $g(y_j)=\dfrac{1}{\sqrt{2\pi}}e^{-\frac{1}{2}(y_j-8)^2}$. \\
Let $h(Y_j)$ be the indicator that whether $Y_j>8$. \\

So with monty carlo method, we can get that \\
$$c=P(Y>8)=\E[\I(Y>8)]=\dfrac{1}{n}\sum_{j=1}^n\I(Y_j'>8)=\dfrac{1}{n}\sum_{j=1}^nh(Y_j')$$
where $Y_j'\sim \N(0,1)$.\\

With importance sampling, since
$$I=\E_f[h(Y)]=\int h(y)f(y)dy=\int\dfrac{h(y)f(y)}{g(y)}g(y)dy=E_g\left[\dfrac{h(Y)f(Y)}{g(Y)}\right]$$
So $I'=\dfrac{1}{n}\sum\limits_{i=1}^n\dfrac{h(Y_j)f(Y_j)}{g(Y_j)}$, where $Y_j\sim \N(8,1)$.\\
we can get that
\begin{align*}
c' &= \dfrac{1}{n}\sum_{j=1}^n\dfrac{h(Y_j)f(Y_j)}{g(Y_j)} \\
&= \dfrac{1}{n}\sum_{j=1}^n\I(Y_j>8)\cdot \dfrac{\frac{1}{\sqrt{2\pi}}e^{-\frac{1}{2}Y_j^2}}{\frac{1}{\sqrt{2\pi}}e^{-\frac{1}{2}(Y_j-8)^2}} \\
&= \dfrac{1}{n}\sum_{j=1}^n\I(Y_j>8)\cdot e^{-8Y_j+32}
\end{align*}

So we just need to sample $n$ times, $Y_j\sim \N(0,8)$, and calculate the average of $\I(Y_j>8)\cdot e^{-8Y_j+32}$.\\

The result using important sampling method is $6.252836307280847e-16$. \\
Which is very close to the correct answer $6.25*10^{-16}$. \\
So we can regard that the importance sampling method is effective and provide correct answers. \\

The following are the pages from jupyter notebook to show that the code can successful run out the images and calculation results we mentioned above. \\



3. The pros and cons of importance sampling. \\
Pros of Importance Sampling:
\begin{itemize}
    \item It is efficient because it allows sampling from a distribution that is easier to sample from, which simplifies the sampling process.
    \item It improves accuracy for rare events, provides more accurate estimates when dealing with low-probability tail.
\end{itemize}
Cons of Importance Sampling:
\begin{itemize}
    \item It depends on the choice of the proposal distribution. A poor choice can lead to high variance and inaccurate results.
    \item It requires more computations to calculate importance weights.
\end{itemize}

\end{homeworkProblem}

\newpage